\documentclass{beamer}
\usetheme{Pittsburgh}

\usepackage[utf8]{inputenc}
\usepackage{default}
\usepackage[procnames]{listings}
\usepackage{graphicx}
%\usepackage[toc,page]{appendix}
\usepackage{caption}
\usepackage{hyperref}
\usepackage{color}
\usepackage{ulem}
%\usepackage{minted}


%Python
\definecolor{keywords}{RGB}{255,0,90}
\definecolor{comments}{RGB}{0,0,113}
\definecolor{red}{RGB}{160,0,0}
\definecolor{green}{RGB}{0,150,0}
\lstset{language=Python, 
    basicstyle=\ttfamily\scriptsize, 
    keywordstyle=\color{keywords},
    commentstyle=\color{comments},
    stringstyle=\color{red},
    identifierstyle=\color{green},
    procnamekeys={def,class},
    breaklines=true,
    columns=fullflexible,
    %Numbering and Tabs
    tabsize=4,
    showspaces=false,
    showstringspaces=false}

\begin{document}

\title{Python 2.7.6}
\subtitle{Object-Oriented Programming Concepts, Non-Primitive Data Types, and Generics}
\author{
  Mallick, Arka\\
  Naazare, Menaka \\
  Nair, Deebul\\
  Quignon, Christophe \\
} 
\institute{Hochschule Bonn Rhein Sieg}
\date{\today}

\begin{frame}
\titlepage
\end{frame}

%
\begin{frame}[fragile]
\frametitle{Threads and Processes}

\end{frame}

%
\begin{frame}[fragile]
\frametitle{Sharing Information Between Threads, Thread Communication and Synchronization}

\end{frame}

%
\begin{frame}[fragile]
\frametitle{Inter-Process Communication and Networking}

\end{frame}

%\begin{frame}
% \frametitle{references}
% \begin{itemize}
%  \item Mark Lutz. 2003. Learning Python (2 ed.). O'Reilly \& Associates, Inc., Sebastopol, CA, USA.
%  \item \href{https://docs.python.org/2.7/}{docs.python.org/2.7/}
%  \item \href{http://interactivepython.org}{interactivepython.org}
% \end{itemize}
%\end{frame}
	

%COPY AND PASTE FROM HERE

%\begin{enumerate}
% \item 
%\end{enumerate}

%\hyperref{link}{text}

%\begin[Language=Python]{lstlisting}
%#PYTHON CODE HERE
%\end{lstlisting}

%\lstinputlisting[Language=Python]{ }

%\begin{figure}
% \center
% \includegraphics[width= cm]{img/ }
% \caption{}
%\end{figure}

%BIBLIOGRPAHY?
%\bibliographystyle{ieee}%amsalpha
%\bibliography{ }


\end{document}
