\documentclass{beamer}
\usetheme{Pittsburgh}

\usepackage[utf8]{inputenc}
\usepackage[procnames]{listings}
\usepackage{listings}
\usepackage{graphicx}
%\usepackage[toc,page]{appendix}
\usepackage{caption}
\usepackage{hyperref}
\usepackage{color}
\usepackage{ulem}
%\usepackage{minted}

%Python
\definecolor{keywords}{RGB}{255,0,90}
\definecolor{comments}{RGB}{0,0,113}
\definecolor{red}{RGB}{160,0,0}
\definecolor{green}{RGB}{0,150,0}
\lstset{language=Python, 
    basicstyle=\ttfamily\scriptsize, 
    keywordstyle=\color{keywords},
    commentstyle=\color{comments},
    stringstyle=\color{red},
    identifierstyle=\color{green},
    procnamekeys={def,class},
    breaklines=true,
    columns=fullflexible,
    %Numbering and Tabs
    %numbers=left,
    %tabsize=4,
    %showspaces=false,
    %showstringspaces=false
    }

\begin{document}

\title{Python 2.7.6}
\subtitle{Threading and Networking}
\author{
  Mallick, Arka\\
  Naazare, Menaka \\
  Nair, Deebul\\
  Quignon, Christophe \\
} 
\institute{Hochschule Bonn Rhein Sieg}
\date{\today}

\begin{frame}
\titlepage
\end{frame}

%Chris
\begin{frame}[fragile]
\frametitle{Thread}
Withs thread, multiple parts of the program can run interleaved. Threads are linked to their programm and share memory and state with it.
The threads will not leave the CPU and thus will not decrease the runtime of your program. But are executed in "parallel" (more correct: interleaved).


\begin{lstlisting}[language=Python]
import threading

thread.start_new_thread(function, args[, kwargs])

lock.acquire([waitflag])
lock.release()
\end{lstlisting}
\end{frame}

\begin{frame}[fragile]
\frametitle{Processes}
A process is an own instance of a programm. They are totally independent and thus can exploit multiple CPUs. They can may communicate. 

\begin{lstlisting}[language=Python]
from multiprocessing import Process

p = Process(target=function, args="none")
p.start()
p.join()
\end{lstlisting}
\end{frame}


%Arka
\begin{frame}[fragile]
\frametitle{Sharing Information Between Threads}

\end{frame}


%Menaka
\begin{frame}[fragile]
\frametitle{Thread Communication and Synchronization}

\end{frame}


%Deebul
\begin{frame}[fragile]
\frametitle{Inter-Process Communication and Networking}

\end{frame}

\begin{frame}
 \frametitle{references}
 \begin{itemize}
  \item Mark Lutz. 2001. Programming Python (2 ed.). O'Reilly \& Associates, Inc., Sebastopol, CA, USA.
  \item \href{https://docs.python.org/2.7/}{docs.python.org/2.7/}
  \item \href{http://en.wikibooks.org/wiki/Python_Programming/Threadingfrom}{en.wikibooks.org/wiki/Python\_Programming/Threadingfrom}
 \end{itemize}
\end{frame}
	

%COPY AND PASTE FROM HERE

%\begin{enumerate}
% \item 
%\end{enumerate}

%\hyperref{link}{text}

%\begin[Language=Python]{lstlisting}
%#PYTHON CODE HERE
%\end{lstlisting}

%\lstinputlisting{ }[Language=Python]

%\begin{figure}
% \center
% \includegraphics[width= cm]{img/ }
% \caption{}
%\end{figure}

%BIBLIOGRPAHY?
%\bibliographystyle{ieee}%amsalpha
%\bibliography{ }


\end{document}
