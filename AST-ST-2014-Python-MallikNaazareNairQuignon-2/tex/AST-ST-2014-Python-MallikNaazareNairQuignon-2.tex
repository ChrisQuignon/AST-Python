\documentclass{beamer}
\usetheme{Pittsburgh}

\usepackage[utf8]{inputenc}
\usepackage{default}
\usepackage[procnames]{listings}
\usepackage{graphicx}
%\usepackage[toc,page]{appendix}
\usepackage{caption}
\usepackage{hyperref}
\usepackage{color}

%Python
\definecolor{keywords}{RGB}{255,0,90}
\definecolor{comments}{RGB}{0,0,113}
\definecolor{red}{RGB}{160,0,0}
\definecolor{green}{RGB}{0,150,0}
\lstset{language=Python, 
    basicstyle=\ttfamily\scriptsize, 
    keywordstyle=\color{keywords},
    commentstyle=\color{comments},
    stringstyle=\color{red},
    identifierstyle=\color{green},
    procnamekeys={def,class},
    breaklines=true,
    columns=fullflexible,
    %Numbering and Tabs
    tabsize=4,
    showspaces=false,
    showstringspaces=false}

\begin{document}

\title{Python 2.7.6}
\subtitle{Object-Oriented Programming Concepts, Non-Primitive Data Types, and Generics}
\author{
  Mallick, Arka\\
  Naazare, Menaka \\
  Nair, Deebul\\
  Quignon, Christophe \\
  %Familyname, Name \texttt{email}
} 
\institute{Hochschule Bonn Rhein Sieg}
\date{\today}

\begin{frame}
\titlepage
\end{frame}

%CONTENTS
%Please produce a set of summary slides surveying the programming language concepts concerning the following topics:
%
%    1 classes, instances, objects
%    2 attributes/fields, instance variables, class variables
%    3 methods, arguments lists, return types, modifiers, variable arity methods, defaults
%    ARKA 4 inheritance and overriding, polymorphism
%    DEBUL 5 non-primitive data types: vectors, arrays, lists, enumerations, sets, maps, trees, graphs
%    DEBUL 6 generic data types
%
%
%The task is to be solved in group in class.
%
%Produce your results in presentable form, e.g. as a set of slides accompanied by running program examples.
%
%Put the presentation in a directory named using the following scheme: "AST-ST-2014-pl-LastNames-2", where you replace pl by the language you took care of and LastName by a camel cased concatenation of the last names of your team members.
%Add in this directory the program examples, each in a separate subdirectory. Make sure all your source looks impeccable and follows the coding standards for the language. Then zip the directory and upload here.

\begin{frame}[fragile]
\frametitle{rediara}
\framesubtitle{}

\end{frame}


%\begin{frame}
% \frametitle{references}
% \begin{itemize}
%  \item Mark Lutz. 2003. Learning Python (2 ed.). O'Reilly \& Associates, Inc., Sebastopol, CA, USA.
%  \item \href{https://docs.python.org/2.7/}{docs.python.org/2.7/}
%  \item \href{http://interactivepython.org}{interactivepython.org}
% \end{itemize}
%\end{frame}

%COPY AND PASTE FROM HERE

%\begin{enumerate}
% \item 
%\end{enumerate}

%\hyperref{link}{text}

%\begin[Language=Python]{lstlisting}
%#PYTHON CODE HERE
%\end{lstlisting}

%\lstinputlisting[Language=Python]{ }

%\begin{figure}
% \center
% \includegraphics[width= cm]{img/ }
% \caption{}
%\end{figure}

%BIBLIOGRPAHY?
%\bibliographystyle{ieee}%amsalpha
%\bibliography{ }


\end{document}
