\documentclass{beamer}
\usetheme{Pittsburgh}

\usepackage[utf8]{inputenc}
\usepackage{default}
\usepackage[procnames]{listings}
\usepackage{graphicx}
%\usepackage[toc,page]{appendix}
\usepackage{caption}
\usepackage{hyperref}
\usepackage{color}
\usepackage{ulem}
%\usepackage{minted}


%Python
\definecolor{keywords}{RGB}{255,0,90}
\definecolor{comments}{RGB}{0,0,113}
\definecolor{red}{RGB}{160,0,0}
\definecolor{green}{RGB}{0,150,0}
\lstset{language=Python, 
    basicstyle=\ttfamily\scriptsize, 
    keywordstyle=\color{keywords},
    commentstyle=\color{comments},
    stringstyle=\color{red},
    identifierstyle=\color{green},
    procnamekeys={def,class},
    breaklines=true,
    columns=fullflexible,
    %Numbering and Tabs
    tabsize=4,
    showspaces=false,
    showstringspaces=false}

\begin{document}

\title{Python 2.7.6}
\subtitle{QA}
\author{
  Mallick, Arka\\
  Naazare, Menaka \\
  Nair, Deebul\\
  Quignon, Christophe \\
} 
\institute{Hochschule Bonn Rhein Sieg}
\date{\today}

\begin{frame}
\titlepage
\end{frame}


\begin{frame}[fragile]
\frametitle{How to use the "keyword arguments"?}
By input order.
\begin{lstlisting}[language=Python]
def foo(*positional, **keywords):
    print "Positional:", positional
    print "Keywords:", keywords
\end{lstlisting}

\begin{lstlisting}[language=Python]
foo('one','two',c='three',d='four')
Positional: ('one', 'two')
Keywords: {'c': 'three', 'd': 'four'}
\end{lstlisting}
\begin{lstlisting}[language=Python]
 foo(a='one', b='two', 'three', 'four')
SyntaxError: non-keyword arg after keyword arg
\end{lstlisting}
\hbox{}
\hbox{}
\scriptsize
\hfill{}\href{http://stackoverflow.com/questions/1419046/}{stackoverflow.com/questions/1419046/}

\end{frame}


\begin{frame}[fragile]
\frametitle{How does "sort" work for a list with different data types?}

\end{frame}


\begin{frame}[fragile]
\frametitle{What are hashable and non-hashable objects?}
An object is hashable if it has a hash value which never changes during its lifetime (it needs a \_\_hash\_\_() method), 
and can be compared to other objects (it needs an \_\_eq\_\_() or \_\_cmp\_\_() method). 
Hashable objects which compare equal must have the same hash value. 
Hashability makes an object usable as a dictionary key and a set member, because these data structures use the hash value internally. 
All of Python’s immutable built-in objects are hashable, while no mutable containers (such as lists or dictionaries) are. 
Objects which are instances of user-defined classes are hashable by default; they all compare unequal (except with themselves), 
and their hash value is their id().
 
\end{frame}

\begin{frame}[fragile]
\frametitle{Non-primitive data types}
\begin{itemize}
	\item Non-primitive data types are not defined by the programming language, but are instead created by the programmer. They are sometimes called "reference variables," or "object references," since they reference a memory location, which stores the data. 
While an object may contain any type of data, 
the information referenced by the object may still be stored as a primitive data type.
	\item The fact that there is a non-primitive data-type implicites that their acutal values lie in the Heap, and there's only a reference to them in the runtime-stack. 
\end{itemize}
\end{frame}


\begin{frame}[fragile]
\frametitle{Non-primitive data types}
Example of Non-Primitive data types:
\begin{enumerate}
	\item \href{http://www.python-course.eu/sequential_data_types.php}{Sequential Data Types}
	  \begin{enumerate}
	    \item String
	    \item Lists
	    \item Tuples
	  \end{enumerate}
	\item \href{http://www.python-course.eu/dictionaries.php}{Dictionaries}
	\item \href{http://www.python-course.eu/sets_frozensets.php}{Sets and Frozensets}\ldots

\end{enumerate}

\end{frame}

%\begin{frame}
% \frametitle{references}
% \begin{itemize}
%  \item Mark Lutz. 2003. Learning Python (2 ed.). O'Reilly \& Associates, Inc., Sebastopol, CA, USA.
%  \item \href{https://docs.python.org/2.7/}{docs.python.org/2.7/}
%  \item \href{http://interactivepython.org}{interactivepython.org}
% \end{itemize}
%\end{frame}
	

%COPY AND PASTE FROM HERE

%\begin{enumerate}
% \item 
%\end{enumerate}

%\hyperref{link}{text}

%\begin[Language=Python]{lstlisting}
%#PYTHON CODE HERE
%\end{lstlisting}

%\lstinputlisting[Language=Python]{ }

%\begin{figure}
% \center
% \includegraphics[width= cm]{img/ }
% \caption{}
%\end{figure}

%BIBLIOGRPAHY?
%\bibliographystyle{ieee}%amsalpha
%\bibliography{ }


\end{document}
